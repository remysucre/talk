\documentclass{beamer}

% \usepackage{biblatex}
% \addbibresource{references.bib}

\usepackage[small,nohug,heads=vee]{diagrams}
\diagramstyle[labelstyle=\scriptstyle]
\usepackage{caption}
\usepackage{subcaption}
\usepackage{mathtools}
\usepackage{listings}
\lstset{
  columns=flexible,
  mathescape=true,
  keepspaces=true,
  showstringspaces=false,
  stringstyle=\slshape\color{green!40!black},
  basicstyle=\ttfamily\small,
  language=SQL,
  morekeywords={*, self},
  commentstyle=\slshape\color{black!60},
  tabsize=2,
}

\newcommand{\B}{\mathbb B} % complex numbers
\newcommand{\C}{\mathbb C} % complex numbers
\newcommand{\N}{\mathbb N} % the natural numbers
\newcommand{\R}{\mathbb R} % the real numbers
\newcommand{\Q}{\mathbb Q} % the rational numbers
\newcommand{\D}{\mathbf D} % bold-face D, used for generic domain
\newcommand{\SR}{\mathbf S} % semiring
\newcommand{\dleq}{\mbox{ :- }}
\newcommand{\deq}{\stackrel{\text{def}}{=}}

\newcommand{\set}[1]{\{#1\}}                    % Set (as in \set{1,2,3}).
\newcommand{\setof}[2]{\{{#1}\mid{#2}\}}        % Set (as in \setof{x}{x>0}).

\usetheme{metropolis}           % Use metropolis theme
\setbeameroption{show notes}
% \setbeameroption{show only notes}
% \setbeameroption{hide notes}

\title{Optimizing Modern Data Processing Systems with Automated Reasoning}
\date{\today}
\author{Remy Wang}
\institute{University of Washington}
\begin{document}
  \maketitle
  \note{Welcome. My name is Remy and today I will talk about how to optimize modern data processing 
  systems with automated reasoning.}

  \begin{frame}{Overview}
    \note{Here is an overview of my talk. 
    I will start by introducing various challenges posed by modern data processing, 
    then cover some background on query Optimization and reasoning.
    Next I will present the functional representation of data 
    and show you how to build optimizers that reason. 
    Finally, I will propose a new research project to improve optimizers with reasoning 
    before concluding.}
    \tableofcontents
  \end{frame}

  \section{Modern Data Processing}
  \note{Let's get started.}

  \begin{frame}{Modern Data Processing: Queries}
    \note{Here is an example query that may be part of a modern data analytics task. 
    It computes the Betweenness centrality of a graph E.
    You don't need to understand the details about the computation,
    and I will just highlight a few aspects of the query.
    First we see that the query operates on a graph, which may be represented by a matrix, 
    instead of relational data found in traditional databases.
    Second, the computation is driven mainly by aggregation and arithmetic like multiplication and division.
    Third, the query is recursive, as both the body and the head contains C. 
    Even though some of these features can be modeled relationally, they are not well-supported 
    by traditional optimizers. 
    For example, here is a very simple optimization rewriting the count of a cartesian product into 
    the product of counts; yet Postgres cannot perform the optimization. 
    }
    Betweenness centrality of a graph $E$:
    \begin{align*}
      C(s, v) = \sum_{t: E(v, t) \wedge D(s, t)=D(s, v+1)} 
      \frac{\sigma(s, v)}{\sigma(s, t)}(1+ C(s, t))
    \end{align*}
    where $D$ is the distance, $\sigma$ is the \# of shortest paths. \pause
    \begin{itemize}
      \item Non-relational data \pause
      \item Aggregation \& interpreted functions \pause
      \item Recursive \pause
     \end{itemize}
    \textbf{Expressiveness} not well-supported by traditional optimizers:
    \[\#[R \times S] \Rightarrow \#[R] \cdot \#[S]\]
  \end{frame}

  \begin{frame}{Modern Data Processing: Data}
    \note{Apart from queries, the other side of data processing is data. 
    It is well-known that the world is generating more and more data at an increasing speed.
    If we want to analyze such large and fast data efficiently, 
    the optimizer must produce faster query plans using shorter time.
    }
    \begin{columns}
      \begin{column}{{0.5\textwidth}}
        \includegraphics[width=\linewidth]{datasphere.png}
      \end{column}
      \begin{column}{{0.5\textwidth}}
        \includegraphics[width=\linewidth]{fb.png}
      \end{column}
    \end{columns}
    \begin{itemize}
      \item Data is increasing in \textbf{volume} \& \textbf{velocity} \pause
      \item The optimizer needs to produce faster plans in shorter time
    \end{itemize}
  \end{frame}

  \begin{frame}{Modern Data Processing: Systems}
    \note{Another dimension of data processing is the systems themselves.
    Recent years have seen a Cambrian explosion of data processing systems. 
    Since each of these systems need to optimize queries, 
    simplifying the design and implementation of optimizers can save a lot of time for everyone. }
    \begin{figure}
    \includegraphics[width=\linewidth]{systems.png}
    \end{figure}
    \begin{itemize}
      \item Every data processing system needs an optimizer!
      \item Simplifying optimizers can save a lot of time for everyone.
    \end{itemize}
  \end{frame}

  \begin{frame}{My Theses}
    \note{My goal is exactly to improve query optimizers to support modern data processing. 
    My these are the following:}
    \begin{itemize}
      \item Core components and techniques from \textbf{automated reasoning}
      can make query optimizers simpler, more efficient,
      and more effective. \pause
      \item The \textbf{functional representation} of data can enable query 
      optimizers to effectively leverage automated reasoning tools.
    \end{itemize}
  \end{frame}

  \section{Background: Query Optimization and Reasoning}
  \note{Before I present the techniques, let's review some background on query optimization and reasoning.}

  \begin{frame}{Query Optimization}
    \note{Here is a cartoon illustration of query optimization. 
    The optimizer searches among a space of query plans, 
    where a part of the space are those equivalent to the input.
    Among these, the optimizer tries to find a plan that is more efficient.
    We can think of this as a synthesis problem, where the optimizer 
    needs to synthesize a query that satisfies the specification 
    that it is both equivalent to the input and faster.}
    \begin{figure}
      \includegraphics[width=0.7\linewidth]{opt.pdf}
    \end{figure}
    Given an input query and constraints on the data, 
    output an equivalent query that is more efficient. \pause \\
    This is a \textbf{synthesis problem}: given a specification,
    output a program that satisfies the spec.
  \end{frame}

  \begin{frame}{Query Optimization}
    \note{There is a problem related to query optimization: ... 
    This is a verification problem.}
    A related problem: given constraints ($\Sigma$) on the data ($I$), 
    check if two queries ($\psi, \phi$) are equivalent.
    \[\forall I . \Sigma(I) \implies (\psi(I) = \phi(I)) \] \pause
    This is a \textbf{verification problem}:
    given a specification, check that a program meets the specification.
  \end{frame}

  \begin{frame}{Automated Reasoning}
    \note{The good news is, ..., and some have already been applied 
    to improve data processing}
    Both verification problems and synthesis problems are studied 
    extensively in automated reasoning.
  \end{frame}

  \begin{frame}{Automated Reasoning for Data Processing}
    \note{One example of applying reasoning to data processing 
    involves the well-known chase procedure. ...}
    \textbf{The Chase} is an algorithm that has wide applications 
    in data management. It has the property:
    \begin{alertblock}{The Chase preserves query equivalence}
      Chasing a query $Q$ with a set of integrity constraints $\Sigma$
      results in a query $Q'$ s.t.~$\Sigma \models Q'\equiv Q$.
    \end{alertblock}
  \end{frame}

  \begin{frame}{Automated Reasoning for Data Processing}
    \textbf{Chase \& Backchase}~\cite{backchase}: Given $Q$, find the smallest~\footnote{
      Refer to~\cite{backchase} for a formal definition of ``small''.
    }
    query $Q'$ s.t.~$Q'\equiv Q$.
    \begin{figure}
      \includegraphics[width=0.3\linewidth]{backchase.pdf}
    \end{figure}
    First chase the input into a {\em universal} plan $\hat Q$, \pause 
    then check if any subplan $Q'$ of $\hat Q$ is equiv.~to $Q$,
    by chasing $Q'$.
  \end{frame}

  % \begin{frame}{Automated Reasoning for Data Processing}
  %   \textbf{Tree Automata} can capture certain class of
  %   relational queries, by coding tree-like data with trees.
  %   \begin{figure}
  %     \colorbox{white}{
  %     \includegraphics[width=0.3\linewidth]{tree1.pdf}
  %     \hspace{1cm}
  %     \includegraphics[width=0.3\linewidth]{tree2.pdf}}
  %   \end{figure}
  %   \pause
  %   Emptiness of tree automata is \textbf{decidable}! \\
  %   Check query containment: 
  %   $Q_1 \subseteq Q_2 \rightarrow A_{Q_1 \wedge \neg Q_2} = \emptyset$
  % \end{frame}

  \begin{frame}{Automated Reasoning for Data Processing}
    \textbf{Axiomatic Rewriting} proves equivalence by chaining together 
    a sequence of axioms. \pause
    Rule-based query optimizers essentially implement that, for example:
    \[\forall R, S, p: \sigma_p (R\bowtie S) \equiv \sigma_p(R) \bowtie \sigma_p(S)\]
    corresponds to the ``push-down selection'' rewrite.\pause

    If two queries can rewrite to each other, they are equivalent.
  \end{frame}

  \begin{frame}{Automated Reasoning for Data Processing}
    \textbf{Program Verification} checks if a program satisfies a specification:
    \[\Psi(Q_{\text{in}}, Q) \stackrel{def}{=} \forall \bar R : Q_{\text{in}}(\bar R) = Q(\bar R)\]
    which is valid if $Q_{\text{in}}(\bar R) \neq Q(\bar R)$ is UNSAT (no model for $\bar R$). \pause

    \textbf{Program Synthesis} generates a program satisfying a specification:
    \[\Phi(Q_{\text{in}}) \stackrel{def}{=} \forall \bar R : Q_{\text{in}}(\bar R) = Q(\bar R)\]
    which gives a solution when there is a model for $Q$.
  \end{frame}

  \begin{frame}{Automated Reasoning for Data Processing}
    \textbf{Goal}: leverage automated reasoning to build query optimizers.

    \textbf{Approach}: functional representation + axiomatic rewrite + SMT
  \end{frame}
  
  \section{Functional Representation of Data}
  \begin{frame}{The Functional Representation of Data}
    Generalize every relation $R : \D^a$ to a function 
    $f_R : \D^a \rightarrow \SR$,
    where $\SR$ is a commutative semiring~\cite{semiring}. \pause
    \begin{itemize}
      \item A $\B$-relation is a standard relation under \textbf{set} semantics. \pause
      \item An $\N$-relation is a standard relation under \textbf{bag} semantics. \pause
      \item When $\D = [n]$, an $\R$-relation is a tensor over the reals.
    \end{itemize}
  \end{frame}

  \begin{frame}{The Functional Representation of Data}
    An expression in $\SR$-relational algebra is one of:
    \begin{enumerate}
      \item An \textbf{atom} $r$ of the form $R(x_1, ..., x_a)$.
      \item A \textbf{natural join} of two expressions $e_1 \times e_2$.
      \item A \textbf{union} of two expressions, $e_1 + e_2$.
      \item An \textbf{aggregate},  $\sum_x e$.
    \end{enumerate} \pause
    A variable $x$ is \textbf{bound} in $e$ if $e$ contains $\sum_x e'$;
    otherwise it is \textbf{free}. \pause 
    Every atom $R(x_1, ..., x_a)$ evaluates to some $s \in \SR$, and
    $+, \times$ and $\sum$ compute over $\SR$.
  \end{frame}

  \begin{frame}{The Functional Representation of Data}
    $\sum_y E(x, y) \times E(y, z)$ specializes to 
    different semantics under different semiring $\SR$:
    \begin{enumerate}
      \item Under $(\mathbb{B},\vee, \wedge, \texttt{false}, \texttt{true})$
          it is the binary join $Q(x, z) \dleq \exists y: E(x, y) \wedge E(y, z).$ 
          under set semantics. \pause
      \item Under $(\mathbb{N}, +, \times, 0, 1)$ it is under
          bag semantics. \pause
      \item Under $(\mathbb{R}, +, \times, 0, 1)$ it computes the matrix product $EE$. \pause
      \item Under $(\mathbb{N} \cup \set{\infty}, \min, +, \infty, 0)$ it computes the lengths
          of all shortest 2-paths in a graph:
          $Q(x, z) = \min_y E(x, y) + E(y, z)$
  \end{enumerate}
  \end{frame}

  \begin{frame}{The Functional Representation of Data}
    \textbf{Query Equivalence} Two queries
    $Q_1({\bar R})$, $Q_2({\bar R})$
    over $\SR$-relations ${\bar R}$ with domain $\mathbf{D}$
    are equivalent iff
    they denote the same function:
    \begin{align*}
    Q_1 =_{\SR,\D} Q_2 \iff
    \forall {\bar x}:\D^a, {\bar R} :
    Q_1({\bar R})({\bar x}) =_{\SR}
    Q_2({\bar R})({\bar x})
    \end{align*} \pause
    Write $=_{\SR}$ for equiv.~over all domains;
    $=$ for equiv.~over all domains and codomains.
  \end{frame}

  \begin{frame}{The Functional Representation of Data}
    \alert{Isomorphism Captures Equivalence} ($\bigstar$)~\cite{spores}
    Let $e_1, e_2$ be two  canonical expressions.
    Then the following conditions are equivalent:
    \begin{itemize}
  % \itemsep0em
    \item \label{item:1} $e_1 \equiv e_2$ (they are isomorphic)
    \item \label{item:2} $e_1 = e_2$ (they are equivalent over all semirings)
    \item \label{item:3} $e_1 =_\C e_2$, $e_1 =_\R e_2$, and $e_1 =_\N e_2$
    \item \label{item:6} $e_1 =_{\N, \D} e_2$ for finite $\D$
    s.t. $|\D| \geq \max(|vars(e_1)|, |vars(e_2)|)$.
    \end{itemize} \pause
    We can change equivalence after changing $\SR$ to $\N$, $\R$, or $\C$ (SMT),
    or by canonizing \& checking isomorphism (axiomatic rewrite).
  \end{frame}

  \begin{frame}{The Functional Representation of Data}
    \textbf{Summary} The functional representation of data enables us to:
    \begin{enumerate}
      \item Capture the diverse semantics in modern analytics.
      \item Reason about queries using well-established techniques.
    \end{enumerate}
  \end{frame}
  
  \section{Building Optimizers that Reason}

  \begin{frame}{Building Optimizers that Reason}
    \onslide<1-2>{\textbf{Program Verification} checks if a program satisfies a specification:
    \[\Psi(Q_{\text{in}}, Q) \stackrel{def}{=} \forall \bar R : Q_{\text{in}}(\bar R) = Q(\bar R)\]
    which is valid if $Q_{\text{in}}(\bar R) \neq Q(\bar R)$ is UNSAT (no model for $\bar R$).}

    \onslide<1>{\textbf{Program Synthesis} generates a program satisfying a specification:
    \[\Phi(Q_{\text{in}}) \stackrel{def}{=} \forall \bar R : Q_{\text{in}}(\bar R) = Q(\bar R)\]
    which gives a solution when there is a model for $Q$.}
  \end{frame}

  \begin{frame}{Verifying Equivalence}
    \textbf{Challenges}
    \begin{itemize}
      \item Not all domain types are well-supported ($\N \cup \set{\infty}$, $\R$).
      \item Hard to encode $\sum$; unfolding is unsound \& blows up
    \end{itemize}
  \end{frame}
  
  \begin{frame}{Verifying Equivalence}
    \textbf{Solution} To support domain types: Theorem ($\bigstar$) 
    allows us to switch to the $\N$-semiring to check equivalence. \pause
    
    To verify the commutativity of join, solve for: 
    \[f_R(x, y) \times f_S(y, z) \neq f_S(y, z) \times f_R(x, y)\]
    where $x, y, z$ are natural numbers, $f_R, f_S$ are uninterpreted
    functions of type $\N \times \N \rightarrow \N$. 
  \end{frame}

  \begin{frame}{Verifying Equivalence}
    \textbf{Solution} To encode $\sum$: Introduce an uninterpreted function
    \[f_{\Sigma}:\N\times \N \rightarrow \N\] 
    and encode every aggregation $\sum_x e$ with $f_{\Sigma}(x, e)$. \pause

    \textbf{Soundness}: $f_{\Sigma}(x_1, e_1) \neq f_{\Sigma}(x_2, e_2)$ only
    if $x_1 \neq x_2$ or $e_1 \neq e_2$.\pause

    \textbf{Incompleteness}: cannot prove 
    e.g.~$\sum_x e_1 + \sum_x e_2 = \sum_x e_1 + e_2$.
  \end{frame}

  \begin{frame}{Verifying Equivalence}
    \textbf{Solution} Alternatively, canonize \& check for isomorphism:
    \[Q_1 \xrightarrow[]{\text{canonize}} \equiv \xleftarrow[]{\text{canonize}} Q_2\]
    Complete by ($\bigstar$) for {\em pure} $\SR$-RA expressions; \pause
    augment with SMT to reason about interpreted functions:
    \[Q_1 \xrightarrow[]{\text{canonize}} \text{SMT} \xleftarrow[]{\text{canonize}} Q_2\]
  \end{frame}

  \begin{frame}{Exploring Query Plans}
    \begin{figure}
      \includegraphics[width=0.7\linewidth]{opt.pdf}
    \end{figure}
    2 ways to explore search equivalence space / search efficient space
  \end{frame}

  \begin{frame}{Searching the Equivalence Space}
    Grow a collection of equivalence plans \& find the cheapest. \pause

    An \textbf{E-Graph}~\cite{eqsat} is a compact data structure to store collections 
    of equivalent expressions:
    \begin{figure}
      \includegraphics[width=0.3\linewidth]{fgn.pdf}
    \end{figure}
    Group \& share equiv.~subexpressions: there are $\Omega(N^2)$ expr.~above!
  \end{frame}

  \begin{frame}{Searching the Equivalence Space}
    Grow an e-graph by applying rewrite rules:
    \begin{figure}
      \includegraphics[width=0.3\linewidth]{fgn.pdf}
    \end{figure}
    while preserving \textbf{congruence}: 
    \[\forall f : \bigwedge_{i \in [k]} x_i \equiv y_i \implies f(x_1, \ldots, x_k ) \equiv f(y_1, \ldots, y_k)\]
  \end{frame}

  \begin{frame}{Searching the Equivalence Space}
    Generate a boolean variable per operation ($B_{op}$) and 
    one per equivalence class ($B_{c}$), then solve~\cite{eqsat} :
    \begin{align*}
      Constraints &\deq B_r \wedge \bigwedge_{op} F(op) \wedge \bigwedge_c G(c)
      \\ F(op) &\deq B_{op} \rightarrow \bigwedge_{c \in op.children} B_c \\ G(c)
      &\deq B_c \rightarrow \bigvee_{op \in c.nodes} B_{op}
      \end{align*}
      \[\textbf{minimize} \sum_{op} B_{op} \cdot C_{op} \textbf{ s.t. } Constraints\]
  \end{frame}

  \begin{frame}{Searching the Efficient Space}
    \begin{figure}
      \includegraphics[width=0.7\linewidth]{opt.pdf}
    \end{figure}
    Search among the efficient plans for one equivalent to the input.
  \end{frame}

  \begin{frame}{Searching the Efficient Space}
    \textbf{Goal} To optimize an expensive recursive query:
    \[Q \deq G(F(\cdots F(\bar R))) \text{ (apply $F$ to a fixpoint)}\] \pause
    \textbf{Solution}~\cite{dove} Find a function $H$ such that:
    \[\forall \bar R : H^*(G(\bar R)) = G(F^*(\bar R))\]
    More efficient when $G$ reduces dimension!
  \end{frame}

  \begin{frame}{Searching the Efficient Space}
    Sufficient to find $H$ s.t.~$H\circ G \equiv G \circ F$:
    \begin{diagram}
      X_0   & \rTo^F & X_1 & \rTo^F & X_2 & \ldots & \rTo^F & X_n\\
     \dTo^G  &            & \dTo^G    &            & \dTo^G    &        &            & \dTo^G\\
      Y_0        & \rTo^H & Y_1 & \rTo^H & Y_2 & \ldots & \rTo^H & Y_n
  \end{diagram}  
  \end{frame}

  \begin{frame}{Searching the Efficient Space}
    How to find $H$? ``Generate \& check'':
    \begin{figure}
      \includegraphics[width=0.7\linewidth]{cegis.pdf}
    \end{figure}
    Enumerate all $\SR$-RA expressions until 
    finding $H$ s.t.~$H\circ G \equiv G \circ F$ \pause

    Hopelessly inefficient!
  \end{frame}

  \begin{frame}{Searching the Efficient Space}
    How to find $H$? Counterexample-guided Synthesis (CEGIS)~\cite{sketch}:
    \begin{figure}
      \includegraphics[width=0.7\linewidth]{cegis.pdf}
    \end{figure}
    Implement both the generator \& checker with an SMT solver \\
    Guide generation of plans by {\em counterexamples} from the checker
  \end{frame}

  \begin{frame}{Searching the Efficient Space}
    \textbf{Challenge} CEGIS requires the checker \& the generator to 
    be implemented entirely in SMT. \pause \\
    But we'd like to combine SMT with rewriting:
    \[Q_1 \xrightarrow[]{\text{canonize}} \text{SMT} \xleftarrow[]{\text{canonize}} Q_2\]\pause
    \textbf{Solution}~\cite{dove} Canonize input query $G \circ F$ before CEGIS,
    then constrain grammar to only generate {\em canonical} plans that 
    can be ``denormalized'' into $H\circ G$.
    \[Q_1 \xrightarrow[]{\text{canonize}} \text{CEGIS} \xrightarrow[]{\text{denormalize}} Q_2\]
  \end{frame}

  \begin{frame}{Building Optimizers that Reason}
    \textbf{Summary}
    \begin{itemize}
      \item Automated reasoning can verify query equivalence.
      \item Both SMT solvers and rewriting can reason about queries.
      \item We can explore the search space with reasoning.
    \end{itemize}
  \end{frame}

  \begin{frame}{Result: Optimizing Linear Algebra}
    VLDB'20~\cite{spores}
    \begin{figure}
        \includegraphics[width=0.6\textwidth]{runtime.pdf}
        \caption*{My optimizer (SPORES) against SystemML}
    \end{figure}
  \end{frame}

  \begin{frame}{Result: Optimizing Linear Algebra}
    VLDB'20~\cite{spores}
    \begin{figure}
      \includegraphics[width=0.7\textwidth]{tf.pdf}
      \caption*{My optimizer (SPORES) against TensorFlow}
    \end{figure}
  \end{frame}

  \begin{frame}{Result: Optimizing Deep Learning Inference}
    MLSys'21~\cite{tensat}
    \begin{figure}
      \includegraphics[width=0.6\textwidth]{all_speedup.pdf}
      \caption*{My optimizer (Tensat) against TASO, inference speedup}
    \end{figure}
  \end{frame}

  \begin{frame}{Result: Optimizing Deep Learning Inference}
    MLSys'21~\cite{tensat}
    \begin{figure}
      \includegraphics[width=0.6\textwidth]{all_optim_time.pdf}
      \caption*{My optimizer (Tensat) against TASO, compile time}
    \end{figure}
  \end{frame}
  
  \begin{frame}{Result: Optimizing Recursion}
    SIGMOD'22 (Conditional Accept)
    \begin{figure}
      \begin{subfigure}[b]{0.4\textwidth}
        \centering
        \includegraphics[width=\textwidth]{basic-bd}
        % \caption{BigDatalog}\label{fig:eval:basic:bd}
      \end{subfigure}
      \hfill
      \begin{subfigure}[b]{0.4\textwidth}
        \centering
        \includegraphics[width=\textwidth]{basic-rs}
        % \caption{RecStep}\label{fig:eval:basic:rs}
      \end{subfigure}

      \begin{subfigure}[b]{0.4\textwidth}
        \centering
        \includegraphics[width=\textwidth]{basic-x}
        % \caption{X}\label{fig:eval:basic:x}
      \end{subfigure}
      \caption*{Speedup of the optimized v.s.\ original program.}
    \end{figure}    
  \end{frame}

  \begin{frame}{Result: Optimizing Recursion}
    SIGMOD'22 (Conditional Accept)
    \begin{figure}
      \centering
      \includegraphics[width=\textwidth]{hard_bench}
      \caption*{Runtime increase as a function of the data size.}
  \end{figure} 
  \end{frame}

  \section{Proposal: Improving Optimizers with Reasoning}

  \begin{frame}{Improving Optimizers with Reasoning}
    A developer of a new system can innovate more freely, 
    but a mature system is difficult to change.

    Relational databases a.la.~SQL are highly mature systems; 
    can they benefit from automated reasoning? 
  \end{frame}

  \begin{frame}{The Cascade Framework}
    Developed for SQLServer~\cite{cascades} and implemented in 
    many SOTA query optimizers. \pause

    Apply a set of \textbf{rewrite rules} to grow a collection 
    of \textbf{equivalent plans}, then extract the best. \pause

    Store plans compactly in a {\em memo table}, where equivalent 
    subplans are grouped into {\em memo groups}.
  \end{frame}

  \begin{frame}{The Cascade Framework}
    Almost identical to equality saturation 
    (e-graph $\mapsto$ {\em memo table}, 
    equivalence class $\mapsto$ {\em memo group}):
    \begin{figure}
      \includegraphics[width=0.3\linewidth]{fgn.pdf}
    \end{figure}
    With one crucial difference: eq.~sat.~maintains 
    \textbf{congruence}.
  \end{frame}

  \begin{frame}{Congruence}
    An equivalence relation $=$ is congruent if:
 \[\forall f : \bigwedge_{i \in [k]} x_i = y_i \implies
  f(x_1, \ldots, x_k) = f(y_1, \ldots, y_k)\] \pause
  \begin{figure}
    \includegraphics[width=0.25\linewidth]{cong1.pdf}
  \end{figure}
  \end{frame}

  \begin{frame}{Congruence}
    An equivalence relation $=$ is congruent if:
 \[\forall f : \bigwedge_{i \in [k]} x_i = y_i \implies
  f(x_1, \ldots, x_k) = f(y_1, \ldots, y_k)\] 
  \begin{figure}
    \includegraphics[width=0.25\linewidth]{cong2.pdf}
  \end{figure}
  \end{frame}

  \begin{frame}{Congruence}
    An equivalence relation $=$ is congruent if:
 \[\forall f : \bigwedge_{i \in [k]} x_i = y_i \implies
  f(x_1, \ldots, x_k) = f(y_1, \ldots, y_k)\] 
  \begin{figure}
    \includegraphics[width=0.25\linewidth]{cong3.pdf}
  \end{figure}
  \end{frame}

  \begin{frame}{Congruence}
    An equivalence relation $=$ is congruent if:
 \[\forall f : \bigwedge_{i \in [k]} x_i = y_i \implies
  f(x_1, \ldots, x_k) = f(y_1, \ldots, y_k)\] 
  \begin{figure}
    \includegraphics[width=0.25\linewidth]{cong4.pdf}
  \end{figure}
  \end{frame}

  \begin{frame}{Congruence}
    An equivalence relation $=$ is congruent if:
 \[\forall f : \bigwedge_{i \in [k]} x_i = y_i \implies
  f(x_1, \ldots, x_k) = f(y_1, \ldots, y_k)\] 
  \begin{figure}
    \includegraphics[width=0.4\linewidth]{cong5.pdf}
  \end{figure}
  \end{frame}

  \begin{frame}{Congruence}
    An equivalence relation $=$ is congruent if:
 \[\forall f : \bigwedge_{i \in [k]} x_i = y_i \implies
  f(x_1, \ldots, x_k) = f(y_1, \ldots, y_k)\] 
  \begin{figure}
    \includegraphics[width=0.4\linewidth]{cong6.pdf}
  \end{figure}
  \end{frame}

  \begin{frame}{Congruence}
    An equivalence relation $=$ is congruent if:
 \[\forall f : \bigwedge_{i \in [k]} x_i = y_i \implies
  f(x_1, \ldots, x_k) = f(y_1, \ldots, y_k)\] 
  \begin{figure}
    \includegraphics[width=0.5\linewidth]{cong7.pdf}
  \end{figure}
  \end{frame}

  \begin{frame}{Congruence}
    An equivalence relation $=$ is congruent if:
 \[\forall f : \bigwedge_{i \in [k]} x_i = y_i \implies
  f(x_1, \ldots, x_k) = f(y_1, \ldots, y_k)\] 
  \begin{figure}
    \includegraphics[width=0.5\linewidth]{cong8.pdf}
  \end{figure}
  \end{frame}

  \begin{frame}{Congruence}
    An equivalence relation $=$ is congruent if:
 \[\forall f : \bigwedge_{i \in [k]} x_i = y_i \implies
  f(x_1, \ldots, x_k) = f(y_1, \ldots, y_k)\] 
  \begin{figure}
    \includegraphics[width=0.5\linewidth]{cong9.pdf}
  \end{figure}
  \end{frame}

  \begin{frame}{Congruence}
    Maintaining congruence is beneficial:
    \begin{itemize}
      \item It can make the data structure more compact 
      \item Which allows exploration to run longer \pause
    \end{itemize}
    Maintaining congruence is non-trivial:
    \begin{itemize}
      \item Naive algorithm runs in quadratic time~\cite{cong} 
      \item Need to efficiently update related data structure
    \end{itemize}
  \end{frame}

  \begin{frame}{Congruence in Cascade-style Optimizers}
    \textbf{Hypothesis}
    An optimizer that maintains congruence can run faster and
    produce more efficient query plans.

    \textbf{Proposal}
    Maintain congruence in Cascade-style optimizers, and investigate
    the impact on the performance of the optimizer \& the 
    plan it produces.
  \end{frame}

  \begin{frame}{Congruence in Cascade-style Optimizers}
    \textbf{Systems to Extend}

    CockroachDB~\cite{cockroachdb} (distributed SQL database):
    \begin{figure}
      \includegraphics[width=0.6\linewidth]{cockroach.jpeg}
    \end{figure}
    Green Plum Orca~\cite{orca} (Modular optimizer):
    \begin{figure}
      \includegraphics[width=0.6\linewidth]{orca.jpeg}
    \end{figure}
    A SQL verifier by Chu et.al.~\cite{chu}
  \end{frame}

  \begin{frame}{Evaluation}
    Investigate impact on optimizer \& query performance.
    \begin{figure}
    \begin{minipage}{0.3\textwidth}
        \includegraphics[width=\textwidth]{perf.pdf} % first figure itself
        \caption*{Compile time}
    \end{minipage}
    \begin{minipage}{0.3\textwidth}
        \includegraphics[width=\textwidth]{perf.pdf} % second figure itself
        \caption*{Compile memory}
    \end{minipage}
    \begin{minipage}{0.3\textwidth}
      \includegraphics[width=\textwidth]{perf.pdf} % second figure itself
      \caption*{Query time}
    \end{minipage}
    \end{figure}
  \end{frame}

  \begin{frame}{Research Plan}
    \textbf{Hack} Congruence can be implemented as a set of rewrite rules:
    \begin{align*}
      e_1 \bowtie e_2 & \rightarrow e_1 \bowtie e_2 \\
      \sigma_p e & \rightarrow \sigma_p e \\
      \pi_{\mathbf{x}} e & \rightarrow \pi_{\mathbf{x}} e \\
      & \cdots
    \end{align*} \pause
    Suboptimal, but achieves same compression \& query performance \pause
    
    Can serve as a proxy / poor person's congruence
  \end{frame}

  \begin{frame}{Research Plan}
    \begin{enumerate}
      \item Implement congruence as rewrites \& measure performance
      \item If congruence seems ``worth it'', implement an optimal algorithm
      \item Measure performance
    \end{enumerate} \pause
    \begin{figure}
      \includegraphics[width=0.3\linewidth]{poor.pdf}
      \caption*{Performance of congruence algorithms}
    \end{figure}
  \end{frame}

  \section{Conclusion}

  \begin{frame}{Conclusion}
    \begin{itemize}
      \item Core components and techniques from \textbf{automated reasoning}
      can make query optimizers simpler, more efficient,
      and more effective. 
      \item The \textbf{functional representation} of data can enable query 
      optimizers to effectively leverage automated reasoning tools.
    \end{itemize}
    \textbf{Completed projects} \\
    An optimizer for linear algebra based on equality saturation\\
    An optimizer for recursive queries based on program synthesis

    \textbf{Proposal} \\
    Extend a Cascade-style optimizer with congruence.
  \end{frame}

  \begin{frame}[allowframebreaks]
    \frametitle{References}
    \bibliographystyle{amsalpha}
    \bibliography{references.bib}
  \end{frame}

  

\end{document}
